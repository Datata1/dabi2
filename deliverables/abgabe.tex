\documentclass[a4paper, 12pt]{article}
\usepackage[utf8]{inputenc}
\usepackage[T1]{fontenc}
\usepackage[ngerman]{babel} % oder die gewünschte Sprache
\usepackage{geometry}
\geometry{a4paper, margin=2.5cm}
\usepackage{amsmath}
\usepackage{amssymb}
\usepackage{graphicx}
\usepackage{fancyhdr}
\pagestyle{empty} % Keine Seitenzahlen auf der Titelseite

\begin{document}

\begin{titlepage}
    \centering
    
    \vspace*{4cm} % Optional: Fügt vertikalen Raum am oberen Rand ein
    
    {\fontsize{24pt}{30pt}\selectfont \textbf{Titel des Projekts}}\par\vspace{1cm}
    
    \hrulefill\par\vspace{0.5cm}
    
    {\Large \textbf{Teilnehmer:}}\par\vspace{0.5cm}
    
    \begin{tabular}{ll}
        Name, Vorname & RZ-Kürzel \\
        \hline
        Wiederstein, Jan-David & wija1025 \\
        Name Teilnehmer 2, Vorname Teilnehmer 2 & Kürzel2 \\
        Name Teilnehmer 3, Vorname Teilnehmer 3 & Kürzel3 \\
        % Füge hier weitere Teilnehmer hinzu
    \end{tabular}\par\vspace{1cm}
    
    \hrulefill\par\vspace{1cm}
    
    {\large Datum der Erstellung: \today}\par
    
    % Optional: Füge hier dein Logo oder das Logo der Institution ein
    % \vspace{2cm}
    % \includegraphics[width=0.3\textwidth]{logo.png}
    
\end{titlepage}

\end{document}